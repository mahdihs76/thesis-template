
% -------------------------------------------------------
%  Abstract
% -------------------------------------------------------


\pagestyle{empty}

\شروع{وسط‌چین}
\مهم{چکیده}
\پایان{وسط‌چین}
\بدون‌تورفتگی
در سال‌هاي اخیر بازار برنامک‌‌هاي گوشی‌هاي هوشمند با گسترش چشمگیري همراه بوده است و کاربران گوشی‌هاي هوشمند براي انجام کارهاي روزمره‌ي خود از این برنامک‌ها بهره میبرند.
یکی از مشکلات اصلی داروخـانـه هـا در مـدیریت انـبار دارو، بـررسی لحـظه‌اي مـدت زمـان مـصرف داروهـاسـت. سـاز و کار سنتی در بـررسی تـاریخ انـقضاي داروهـا در مقیاس بـزرگ کارا نیست. همچنین هـزینه نیروي انـسانی زیادي بـراي ثـبت کردن اطـلاعـات داروهـا بـه صـورت دسـتی و گـزارش گیري صـرف می‌شـود. از این رو بـا بـررسی مـنابـع مـوجـود در داروخـانـه هـا، یک راهکار مبتنی بـر نـرم افـزار و قـابـل اجـرا بـر بسـتر سیستم عـامـل‌هـاي گـوشی‌‌هـاي هـوشـمند بـا عـنوان پـویشگر دارو مـطرح خـواهیم کرد.

\پرش‌بلند
\بدون‌تورفتگی \مهم{کلیدواژه‌ها}: 
داروخانه، برنامک، گوشی هوشمند، تاریخ انقضا، پویشگر، طراحی و توسعه
\صفحه‌جدید
