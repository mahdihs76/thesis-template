
\فصل{چارچوب توسعه برنامک}
باتوجه به این موضوع که این روزه گوشی تلفن همراه هوشمند در دسترس اکثر مردم می‌ باشد، تصمیم گرفتیم تا این پویشگر را در قالب یک برنامک موبایلی (اندروید و آی‌او‌اس) پیاده سازی کنیم. 
یکی از چالش‌هایی که  با آن مواجه بودیم زمان و هزینه توسعه مجزا برای هر یک از سیستم‌عامل‌های موبایل بود که با بررسی‌هایی که انجام دادیم به فلاتر رسیدیم.

\قسمت{فلاتر چیست؟}
چارچوب فلاتر یک چارچوب متن باز برای توسعه برنامک‌های موبایل است. این کتابخانه توسط شرکت گوگل در سال ۲۰۱۷ معرفی شد. فلاتر با زبان برنامه نویسی دارت نوشته شده است، یک زبان چندمنظوره و شی گراء که هنوز آنطور که باید و شاید بر سر زبان‌ها نیفتاده است و محبوبیت زیادی ندارد. جالب است بدانید علی بابا، یکی از بزرگترین شرکت‌های چینی در دنیا به فلاتر اعتماد کرده و برنامک‌های خود را با این چارچوب توسعه داده است. 

\قسمت{توسعه چندسکویی}
می‌دانیم که زبان اصلی برنامه نویسی اندروید، جاوا است و برنامه نویسان آی‌او‌اس هم از سویفت برای توسعه برنامک‌های موبایلی استفاده می‌کنند. جاوا پیچیدگی‌های خاص خودش را دارد و شاید کار با آن به اندازه زبان‌هایی مثل پایتون یا جاوا اسکریپت راحت نباشد. همین مساله باعث شده که سایر زبان‌ها هم برای توسعه برنامه‌های موبایلی به کار گرفته شوند. مثلا ری‌اکت یک راه حل نسبتا آسان برای توسعه برنامک‌های موبایل مبتنی بر جاوا اسکریپت است. یا فلاتر که یک پیاده‌سازی خوب از زبان دارت است.

\قسمت{معماری چارچوب فلاتر}

در این قسمت اصلی ترین بخش‌های معماری چارچوب فلاتر  شده است :

\شروع{فقرات}
\فقره زبان دارت: 
چارچوب فلاتر هسته اصلی خود را مدیون زبان برنامه نویسی دارت می باشد، اگرچه گوگل امروزه از زبان های متعددی در توسعه هرچه بیشتر آن استفاده می کند.

\فقره موتور فلاتر:
این موتور بر پایه زبان C++ نوشته شده و با بهره گیری از کتابخانه های گرافیکی گوگل (Skia) تمامی انیمیشن ها، طرح های گرافیکی و غیره را به طور رندرگیری سطح پایین Rendering) (Low-Level به هسته فلاتر اضافه می کند.

\فقره کتابخانه مبنا:
این کتابخانه که بر اساس زبان دارت نوشته شده است دارای تعداد چشم گیری تابع و کلاس است که به برنامه نویس این امکان را می دهد تا برنامه ها و برنامک‌های متنوعی بسازد.

\فقره افزونه‌های منحصر به فرد طراحی:
چارچوب فلاتر دارای ویجت های بسیاری می باشد که به طور کلی به دو دسته تقسیم می شوند : مورد اول Material Design ها که مورد علاقه اندروید و شرکت گوگل است و مورد دوم ویجت Cupertino که شاکله اصلی طراحی آی‌او‌اس می باشد.

\پایان{فقرات}



\قسمت{لیست‌ها}

برای ایجاد یک لیست‌ می‌توانید از محیط‌های «فقرات» و «شمارش» همانند زیر استفاده کنید.

\begin{multicols}{2}
\شروع{فقرات}
\فقره مورد اول
\فقره مورد دوم
\فقره مورد سوم
\پایان{فقرات}

\شروع{شمارش}
\فقره مورد اول
\فقره مورد دوم
\فقره مورد سوم
\پایان{شمارش}

\end{multicols}


\قسمت{درج شکل}

یکی از روش‌های مناسب برای ایجاد شکل استفاده از نرم‌افزار \لر{LaTeX Draw} و سپس
درج خروجی آن به صورت یک فایل \کد{tex} درون متن 
با استفاده از دستور  \کد{fig} یا \کد{centerfig} است.
شکل~\رجوع{شکل:پوشش رأسی} نمونه‌ای از اشکال ایجادشده با این ابزار را نشان می‌دهد.


\شروع{شکل}[ht]
\centerfig{cover.tex}{.9}
\شرح{یک گراف و پوشش رأسی آن}
\برچسب{شکل:پوشش رأسی}
\پایان{شکل}

\bigskip
همچنین می‌توانید با استفاده از نرم‌افزار \lr{Ipe} شکل‌های خود را مستقیما
به صورت \لر{pdf} ایجاد نموده و آن‌ها را با دستورات \کد{img} یا  \کد{centerimg} 
درون متن درج کنید. برای نمونه، شکل~\رجوع{شکل:گراف جهت‌دار} را ببینید.


\شروع{شکل}[hb]
\centerimg{dag}{7cm}
\شرح{یک گراف جهت‌دار بدون دور}
\برچسب{شکل:گراف جهت‌دار}
\پایان{شکل}


\قسمت{درج جدول}

برای درج جدول می‌توانید با استفاده از دستور  «جدول»
جدول را ایجاد کرده و سپس با دستور  «لوح»  آن را درون متن درج کنید.
برای نمونه جدول~\رجوع{جدول:عملگرهای مقایسه‌ای} را ببینید.


\شروع{لوح}[t]
\تنظیم‌ازوسط

\شروع{جدول}{|c|c|}
\خط‌پر 
\سیاه عملگر & \سیاه عملیات \\ 
\خط‌پر \خط‌پر 
\کد{<} & کوچک‌تر \\ 
\کد{>} & بزرگ‌تر \\ 
\کد{==} &  مساوی \\ 
\کد{<>} & نامساوی \\ 
\خط‌پر
\پایان{جدول}

\شرح{عملگرهای مقایسه‌ای}
\برچسب{جدول:عملگرهای مقایسه‌ای}
\پایان{لوح}



\قسمت{درج الگوریتم}

برای درج الگوریتم می‌توانید از محیط «الگوریتم» همانند زیر استفاده کنید.

\شروع{الگوریتم}{پوشش رأسی حریصانه}
\ورودی گراف $G=(V, E)$
\خروجی یک پوشش رأسی از $G$

\دستور قرار بده $C = \emptyset$  % \توضیحات{مقداردهی اولیه}
\تاوقتی{$E$ تهی نیست}
%\اگر{$|E| > 0$}
%	\دستور{یک کاری انجام بده}
%\پایان‌اگر
\دستور یال دل‌‌خواه $uv \in E$ را انتخاب کن
\دستور رأس‌های $u$ و $v$ را به $C$ اضافه کن
\دستور تمام یال‌های واقع بر $u$ یا $v$ را از $E$ حذف کن
\پایان‌تاوقتی
\دستور $C$ را برگردان
\پایان{الگوریتم}


\قسمت{محیط‌های ویژه}

برای درج مثال‌ها، قضیه‌ها، لم‌ها و نتیجه‌ها به ترتیب از محیط‌های
«مثال»، «قضیه»، «لم» و «نتیجه» استفاده کنید.
برای درج اثبات قضیه‌ها و لم‌ها  از محیط «اثبات» استفاده کنید.

تعریف‌های داخل متن را با استفاده از دستور «مهم» به صورت \مهم{تیره‌} نشان دهید.
تعریف‌های پایه‌ای‌تر را درون محیط «تعریف» قرار دهید.

\شروع{تعریف}[اصل لانه‌کبوتری]
اگر $n+1$ یا بیش‌تر کبوتر درون  $n$ لانه قرار گیرند، آن‌گاه لانه‌ای 
وجود دارد که شامل حداقل دو کبوتر است.
\پایان{تعریف}

